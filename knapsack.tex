% Archivo generado automáticamente por knapsack.c
\documentclass[12pt]{article}
\usepackage[utf8]{inputenc}
\usepackage{graphicx}
\usepackage{array,booktabs}
\usepackage[table]{xcolor}
\usepackage{longtable}
\usepackage{geometry}
\usepackage{pdflscape}
\geometry{margin=0.8in}
\begin{document}
\begin{center}
{\large Instituto Tecnológico de Costa Rica\\[1cm]
\includegraphics[width=0.4\textwidth]{TEC.png}\\[2cm]
{\LARGE \textbf{Proyecto 2: Problema de la Mochila}}\\[2cm]
{\large Investigación de Operaciones\\[2cm]
{\large Profesor: }\\[1cm]
{\large Francisco Jose Torres Roja}\\[2cm]
{\large Integrantes: }\\[1cm]
{\large Jose Pablo Fernandez Jimenez - 2023117752}\\[1cm]
{\large Diego Durán Rodríguez - 2022437509}\\[2cm]
{\large Segundo semestre 2025\\[1cm]
\end{center}
\newpage
\section*{Algoritmo de la Mochila}
El problema de la mochila es un problema clásico de optimización combinatoria en el que se busca seleccionar un subconjunto de objetos de un conjunto dado, cada uno con un peso y un valor, de manera que el valor total de los objetos seleccionados se maximice sin exceder la capacidad de la ``mochila''. Este problema tiene varias versiones según las restricciones sobre la cantidad de veces que un objeto puede ser incluido:

\textbf{0/1 Knapsack (Mochila 0/1):} En esta variante, cada objeto puede ser incluido como máximo una vez. La decisión de incluir o no un objeto es binaria (0 o 1).

\textbf{Bounded Knapsack (Mochila acotada):} Aquí, cada objeto puede ser incluido hasta un número limitado de veces.

\textbf{Unbounded Knapsack (Mochila ilimitada):} En esta versión, no hay límite en la cantidad de veces que un objeto puede ser incluido (cada objeto puede ser incluido infinitas veces).

\bigskip
\section*{Algoritmo utilizado}
Para resolver el problema de la mochila implementamos un algoritmo de programación dinámica basado en la ecuación de Bellman. Este enfoque permite calcular de manera eficiente el valor óptimo de la mochila para cada capacidad y conjunto de objetos posibles.

El procedimiento general es el siguiente:
\begin{enumerate}
  \item Inicializar una tabla donde cada fila representa una capacidad de la mochila y cada columna representa los objetos disponibles.
  \item Para cada capacidad $i$ y cada objeto $j$, se calcula el mejor valor posible considerando todas las cantidades $k$ de ese objeto que se pueden incluir, respetando las restricciones de cantidad (0/1, acotada o ilimitada).
  \item Se almacena el mejor valor encontrado en la tabla, junto con el número de copias utilizadas y si existe un empate entre distintas elecciones.
  \item Una vez completada la tabla, se realiza un backtracking para reconstruir todas las soluciones óptimas posibles.
\end{enumerate}

\bigskip
\section*{Problema}
Capacidad de la mochila: $W = 10$\\
Número de objetos: $n = 3$\\

\textbf{Formulación matemática:}\\
\begin{align*}
\textbf{Maximizar Z = }\\
11\,x_{\text{Botella de agua}} + 7\,x_{\text{Medias}} + 12\,x_{\text{Paquete de galletas}}
\end{align*}
\bigskip\bigskip
\textbf{}\\
\begin{align*}
\textbf{Sujeto a:}\\
4\,x_{\text{Botella de agua}} + 3\,x_{\text{Medias}} + 5\,x_{\text{Paquete de galletas}} &\leq 10
\end{align*}
\bigskip\bigskip
\textbf{}\\
\begin{align*}
x_{\text{Botella de agua}} &\in \mathbb{Z}_{\geq 0}\\
x_{\text{Medias}} &\in \mathbb{Z}_{\geq 0}\\
x_{\text{Paquete de galletas}} &\in \mathbb{Z}_{\geq 0}\\
\end{align*}
\section*{Tabla de trabajo}
Color: {\color{green}verde} = se toma el objeto, {\color{red}rojo} = no se toma, {\color{yellow}amarillo} = empate.\\
\begin{landscape}
\subsection*{Objeto Botella de agua}
\begin{longtable}{cc}
\toprule
Peso & Botella de agua \\
\midrule
0 & \cellcolor{red!20}0~(k=0) \\
1 & \cellcolor{red!20}0~(k=0) \\
2 & \cellcolor{red!20}0~(k=0) \\
3 & \cellcolor{red!20}0~(k=0) \\
4 & \cellcolor{green!40}11~(k=1) \\
5 & \cellcolor{green!40}11~(k=1) \\
6 & \cellcolor{green!40}11~(k=1) \\
7 & \cellcolor{green!40}11~(k=1) \\
8 & \cellcolor{green!40}22~(k=2) \\
9 & \cellcolor{green!40}22~(k=2) \\
10 & \cellcolor{green!40}22~(k=2) \\
\bottomrule
\end{longtable}
\end{landscape}
\begin{landscape}
\subsection*{Objeto Medias}
\begin{longtable}{ccc}
\toprule
Peso & Botella de agua & Medias \\
\midrule
0 & \cellcolor{red!20}0~(k=0) & \cellcolor{red!20}0~(k=0) \\
1 & \cellcolor{red!20}0~(k=0) & \cellcolor{red!20}0~(k=0) \\
2 & \cellcolor{red!20}0~(k=0) & \cellcolor{red!20}0~(k=0) \\
3 & \cellcolor{red!20}0~(k=0) & \cellcolor{green!40}7~(k=1) \\
4 & \cellcolor{green!40}11~(k=1) & \cellcolor{red!20}11~(k=0) \\
5 & \cellcolor{green!40}11~(k=1) & \cellcolor{red!20}11~(k=0) \\
6 & \cellcolor{green!40}11~(k=1) & \cellcolor{green!40}14~(k=2) \\
7 & \cellcolor{green!40}11~(k=1) & \cellcolor{green!40}18~(k=1) \\
8 & \cellcolor{green!40}22~(k=2) & \cellcolor{red!20}22~(k=0) \\
9 & \cellcolor{green!40}22~(k=2) & \cellcolor{red!20}22~(k=0) \\
10 & \cellcolor{green!40}22~(k=2) & \cellcolor{green!40}25~(k=2) \\
\bottomrule
\end{longtable}
\end{landscape}
\begin{landscape}
\subsection*{Objeto Paquete de galletas}
\begin{longtable}{cccc}
\toprule
Peso & Botella de agua & Medias & Paquete de galletas \\
\midrule
0 & \cellcolor{red!20}0~(k=0) & \cellcolor{red!20}0~(k=0) & \cellcolor{red!20}0~(k=0) \\
1 & \cellcolor{red!20}0~(k=0) & \cellcolor{red!20}0~(k=0) & \cellcolor{red!20}0~(k=0) \\
2 & \cellcolor{red!20}0~(k=0) & \cellcolor{red!20}0~(k=0) & \cellcolor{red!20}0~(k=0) \\
3 & \cellcolor{red!20}0~(k=0) & \cellcolor{green!40}7~(k=1) & \cellcolor{red!20}7~(k=0) \\
4 & \cellcolor{green!40}11~(k=1) & \cellcolor{red!20}11~(k=0) & \cellcolor{red!20}11~(k=0) \\
5 & \cellcolor{green!40}11~(k=1) & \cellcolor{red!20}11~(k=0) & \cellcolor{green!40}12~(k=1) \\
6 & \cellcolor{green!40}11~(k=1) & \cellcolor{green!40}14~(k=2) & \cellcolor{red!20}14~(k=0) \\
7 & \cellcolor{green!40}11~(k=1) & \cellcolor{green!40}18~(k=1) & \cellcolor{red!20}18~(k=0) \\
8 & \cellcolor{green!40}22~(k=2) & \cellcolor{red!20}22~(k=0) & \cellcolor{red!20}22~(k=0) \\
9 & \cellcolor{green!40}22~(k=2) & \cellcolor{red!20}22~(k=0) & \cellcolor{green!40}23~(k=1) \\
10 & \cellcolor{green!40}22~(k=2) & \cellcolor{green!40}25~(k=2) & \cellcolor{red!20}25~(k=0) \\
\bottomrule
\end{longtable}
\end{landscape}
\section*{Soluciones óptimas}
El valor óptimo es de $Z = 25$ con las siguientes cantidades de cada objeto:\\
\begin{itemize}
\item x_{\text{Botella de agua}} = \textcolor{green}{1}, x_{\text{Medias}} = \textcolor{green}{2}, x_{\text{Paquete de galletas}} = 0\\
\end{itemize}
\renewcommand{\refname}{Referencias}
\begin{thebibliography}{9}
\bibitem{Wikipedia} colaboradores de Wikipedia. (2024, 23 febrero). Problema de la mochila. Wikipedia, la Enciclopedia Libre. \\ \url{https://es.wikipedia.org/wiki/Problema_de_la_mochila}
\end{thebibliography}
\end{document}
